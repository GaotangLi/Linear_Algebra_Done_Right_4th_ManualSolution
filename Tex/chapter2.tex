\documentclass{extarticle}
\sloppy
\input{packages.tex}
\input{math_commands.tex}

\title{\vspace{-2em}Chapter 2: Finite-Dimensional Vector Spaces}
\author{\emph{Linear Algebra Done Right (4th Edition)}, by Sheldon Axler}
\date{Last updated: \today}

\begin{document}
\maketitle 
\tableofcontents
\newpage

\section*{2A: Span and Linear Independence}
\addcontentsline{toc}{section}{2A: Span and Linear Independence}

\begin{definition}[Linear Combination]
    A \emph{linear combination} of a list \(v_1, \ldots, v_m\) of vectors in \(V\)
    is a vector of the form 
    \[a_1 v_1 + \cdots + a_m v_m\]
    where \(a_1, \ldots, a_m \in \F\).
\end{definition}

\begin{definition}[Span]
    The set of all linear combinations of list of vectors \(v_1, \ldots, v_m\) 
    in \(V\) is called the \emph{span} of \(v_1, \ldots, v_m\), denoted by 
    span\((v_1, \ldots, v_m)\). In other words, 
    \[\text{span}(v_1, \ldots, v_m) = \{a_1 v_1 + \cdots a_m v_m \colon a_1, \ldots, a_m \in \F\}\]
    The span of the empty list \(()\) is defined to be \(\{0\}\).
\end{definition}

\begin{thm}
    The span of a list of vectors in \(V\) is the smallest subspace of \(V\) 
    containing all vectors in the list.
\end{thm}

\begin{definition}[Spans]
    If span\((v_1, \ldots, v_m)\) equals \(V\), we say the list \(v_1, \ldots, v_m\)
    \emph{spans} \(V\). 
\end{definition}

\begin{definition}[Finite-dimensional vector space]
    A vector space is called \emph{finite-dimensional} if some list of vectors 
    in it spans the space.
\end{definition}

\begin{definition}[polynomial]
    A function \(p \colon \F \to \F\) is called a \emph{polynomial} with coefficients
    in \(\F\) if there exist \(a_0, \ldots, a_m \in \F\) s.t. 
    \[(z) = a_0 + a_1 z + a_2 z^2 + \cdots + a_m z^m\]
    for all \(z \in \F\).

    \(\Pcal(\F)\) is the set of all polynomials with coefficients in \(\F\).
\end{definition}

\begin{definition}[Linear independence]
    A list of vectors \(v_1, \ldots, v_m\) in \(V\) is called \emph{linearly independent}
    if the only choice of \(a_1, \ldots, a_m \in \F\) that makes 
    \[a_1v_1 + \cdots + a_m v_m = 0\]
    is \(a_1 = \cdots = a_m = 0\). 

    The empty list \(()\) is also declared to be linearly independent.
\end{definition}

\begin{lemma}
    Suppose \(v_1, \ldots v_m\) is a linearly dependent list in \(V\). Then 
    there exists \(k \in \{1, 2, \ldots, m\}\) s.t. 
    \[v_k \in \text{span}(v_1, \ldots, v_{k-1})\]
    Furthermore, if \(k\) satisfies the condition above and the \(k^{\text{th}}\)
    term is removed from \(v_1, \ldots v_m\), then the span of the remaining list 
    equals \(\text{span}(v_1, \ldots, v_m)\).
\end{lemma}

\begin{lemma}[length of linearly independent list \(\leq\) length of spanning list]
    In a finite-dimensional vector space, the length of every linearly independent 
    list of vectors is less than or equal to the length of every spanning 
    list of vectors. 
\end{lemma}

\newpage 
\addcontentsline{toc}{subsection}{2A Problem Sets}
\begin{problem}{1}
    Find a list of four distinct vectors in \(\F^3\) whose span equals 
    \[\{(x, y, z) \in \F^3 \colon x + y + z =0\}\]
\end{problem}

\begin{proof}
Example: \((1,0,-1), (0,-1,1), (1,-1,0), (1,-2,1)\). 
\end{proof}

\begin{problem}{2}
    Prove or give a counterexample: If \(v_1, v_2, v_3, v_4\) spans \(V\), then 
    the list 
    \[v_1 - v_2, v_2 - v_3, v_3 - v_4, v_4\]
    also spans \(V\).
\end{problem}

\begin{proof}
Take any \(v \in V\), then we have \(v = a_1 v_1 + a_2 v_2 + a_3 v_3 
+ a_4 v_4 = a_1(v_1 - v_2) + (a_1 + a_2) (v_2 - v_3) + (a_1 + a_2 + a_3)(
    v_3 - v_4) + (a_1 + a_2 +a_3 + a_4) v_4\). Conversely, any linear combination 
    of these vectors still belong to \(V\).
\end{proof}

\begin{problem}{3}
    Suppose \(v_1, \ldots, v_m\) is a list of vectors in \(V\). 
    For \(k \in \{1, \ldots, m\}\), let 
    \[w_k = v_1 + \cdots + v_k\]
    Show that span(\(v_1, \ldots, v_m\)) = span(\(w_1, \ldots, w_m\))
\end{problem}

\begin{proof}
Take \(v = \sum_{i=1}^m a_i v_i\) from l.h.s, then we can write 

\begin{align*}
    v 
    &= a_1 v_1 + \cdots a_m v_m \\ 
    &= (a_1 - a_m) v_1 +\cdots (a_{m-1} - a_{m}) v_{m-1} + a_m w_m \\ 
    &= (a_1 - a_m - a_{m-1}) v_1 + \cdots (a_{m - 2} - a_{m-1} - a_m) v_{m-2}
    +(a_{m-1} - a_m) w_{m-1}  + a_m w_m \\ 
    &= \sum_{i=1}^m \left(a_i - \sum_{j=i+1}^m a_j \right)w_i \in \text{r.h.s}
\end{align*}

Conversely, take \(w = \sum_{i=1}^m b_i w_i = \sum_{i=1}^m (b_i \sum_{j=1}^i c_j) v_j \in \text{l.h.s.}\).
\end{proof}

\begin{problem}{4}
    (a) Show that a list of length one in a vector space is linearly independent 
    if and only if the vector in the list is not 0. 
    
    (b) Show that a list of length two in a vector space is linearly independent 
    if and only if neither of the two vectors in the list is a scalar multiple of 
    the other.
\end{problem}

\begin{proof}
(a) In order for the only way to write \(a v = 0\) is to ensure \(v \neq 0\). 

(b) \(a v_1 + b v_2 = 0\). \(\Rightarrow\) the only solution is \(a=b=0\) so \(v_1\)
cannot be a multiple of \(v_2\); vice versa.  \(\Leftarrow\) same reason.
\end{proof}

\begin{problem}{8}
    Suppose \(v_1, v_2, v_3, v_4\) is linearly independent in \(V\). Prove that the list 
    \[v_1 - v_2, v_2 - v_3, v_3 - v_4 , v_4\]
    is also linearly independent. 
\end{problem}

\begin{proof}
    \[a_1(v_1 - v_2) + a_2 (v_2 - v_3) + a_3 (v_3 - v_4) + a_4 v_4 = a_1v_1 
    + (-a_1 + a_2)v_2 + (-a_2 + a_3)v_3 + (-a_3 + a_4) v_4\]

    The only solution is \(a_1 = 0, -a_1 + a_2 = 0, -a_2 + a_3 = 0, -a_3 + a_4 = 0\).
\end{proof}

\begin{problem}{9}
    Prove or give a counter example: If \(v_1, \ldots, v_m\) is a linearly independent 
    list of vectors in \(V\), then 
    \[5v_1 - 4v_2, v_2, v_3, \ldots, v_m\]
    is also linearly independent.
\end{problem}

\begin{proof}
\begin{align*}
    &a_1 (5v_1 - 4v_2) + a_2 v_2 + \ldots a_m v_m \\ 
    &=5a_1 v_1 + (-4a_1  + a_2)v_2 + \ldots a_m v_m
\end{align*}

The only solution is \(5a_1 = -4a_1 + a_2 = \ldots = a_m = 0\).
\end{proof}

\begin{problem}{10}
    Prove or give a counterexample: If \(v_1, \ldots, v_m\) is a linearly 
    independent list of vectors in \(V\)  and \(\lambda \in \F\) with 
    \(\lambda \neq 0\), then \(\lambda v_1, \ldots, \lambda v_m\) is linearly 
    independent.
\end{problem}

\begin{proof}
\[\sum_{i}^m \lambda a_i v_i = 0\]

The only solution is \(a_i = 0\) for all \(i\).
\end{proof}

\begin{problem}{12}
    Suppose \(v_1, \ldots, v_m\) is linearly independent in \(V\) 
    and \(w \in V\). Prove that if \(v_1 + w, \ldots, v_m + w\) is 
    linearly dependent, then \(w \in \text{span}(v_1, \ldots, v_m)\).
\end{problem}

\begin{proof}
We know the only solution to \(\sum_i^m a_i v_i\) is all \(a_i = 0\). Now 
we have that non-zero \(a_i\) for solving \(\sum_i^m a_i v_i + \sum_i^m a_i w = 0\).
Thus \(w = \frac{\sum_i^m a_i v_i}{\sum_i^m a_i}\) which completes the proof.  
\end{proof}

\begin{problem}{13}
    Suppose \(v_1, \ldots, v_m\) is linearly independent in \(V\) and 
    \(w \in V\). Show that 
    \[v_1, \ldots, v_m, w \text{ is linearly independent } \Longleftrightarrow 
    w \notin \text{ span}(v_1, \ldots, v_m)\]
\end{problem}

\begin{proof}
\(\Rightarrow\) By contradiction, if \(w\) in the span then it can be written 
as linear combination for some nonzero \(a_i\) and thus they are linearly dependent. 

\(\Leftarrow\) Similarly. 
\end{proof}

\begin{problem}{17}
Prove that \(V\) is infinite-dimensional if and only if there is a sequence 
\(v_1, v_2, \ldots\) of vectors in \(V\) such that \(v_1, \ldots, v_m\) is 
linearly independent for every positive integer \(m\). 
\end{problem}

\begin{proof}
\(\Rightarrow\) There doesn't exist any finite list of vectors that span the space. For 
the sake of contradicting assumes for every sequence  \(v_1, \ldots\) of 
vectors in \(V\) 
\(\exists\) m such that \(v_1, \ldots, v_m\)
is linearly dependent. Then this means we can construct the basis for the vector 
space as follows: select \(v_i \in V\) s.t. \(v_i\) and \(v_1, \ldots, v_{i-1}\)
are linearly independent. This means that there exits \(m\) s.t. \(v_1, \ldots, v_m\)
that spans \(V\), forming a contradiction. 

\(\Leftarrow\) Suppose for contradiction. Then there exists a span, contradicting 
the linear independence claim for every \(m\).
\end{proof}

\begin{problem}{18}
    Prove \(\F^\infty\) is infinite-dimensional. 
\end{problem}

\begin{proof}
We apply Problem 17. Construct \(v_i\) to be the vector that has 1 on the 
\(i\)-th coordinate and 0 elsewhere. clearly \(v_1, v_2, \ldots\) are s.t. 
\(v_1, \ldots, v_m\) is linearly independent for every positive integer \(m\).
\end{proof}

\newpage 
\addcontentsline{toc}{section}{2B: Bases}
\section*{2B: Bases}

\begin{definition}[basis]
    A \emph{basis} of \(V\) is a list of vectors in \(V\) that is linearly 
    independent and spans \(V\).
\end{definition}

\begin{thm}[criterion for basis]
    A list \(v_1, \ldots, v_m\) of vectors in \(V\) is a basis of \(V\)
    if and only if every \(v \in V\) can be written uniquely in the form 
    \[v = \sum_i^m a_i v_i\]
    where \(a_i \in \F\).
\end{thm}

\begin{lemma}[every spanning list contains a basis]
    Every spanning list in a vector space can be reduced to a basis of the 
    vector space.
\end{lemma}

\begin{lemma}
    Every finite-dimensional vector space has a basis.
\end{lemma}

\begin{lemma}
    Every linearly independent list of vectors in a finite-dimensional vector 
    space can be extended to a basis of the vector space. 
\end{lemma}

\begin{lemma}
    Suppose \(V\) is finite-dimensional and \(\Ucal\) is a subspace of \(V\). 
    Then there is a subspace \(W\) of \(V\) such that \(V = \Ucal \oplus W\).
\end{lemma}

\newpage 
\addcontentsline{toc}{subsection}{2B Problem Sets}

\begin{problem}{1}
    Find all vector spaces that have exactly one basis.
\end{problem}

\begin{proof}
The only answer is \(\{0\}\). Otherwise, for any basis \(v\) one can get 
\(av\) for \(a \neq 0, a \neq 1\).
\end{proof}

\begin{problem}{4}
    \begin{enumerate}
        \item Let \(\Ucal\) be the subspace of \(\C^5\) defined by 
        \(\Ucal = \{(z_1, z_2, z_3, z_4, z_5) \in \C^5 \colon 6z_1 = z_2
        , z_3 + 2z_4 + 3z_5 = 0\}\)
        Find a basis of \(\Ucal\).
        \item Extend the basis to a basis in \(\C^5\).
        \item Find a subspace \(\Wcal\) of \(\C^5\) s.t. \(\C^5 = \Ucal \oplus \Wcal\).
    \end{enumerate}
\end{problem}

\begin{proof}
\begin{enumerate}
    \item  \((z_1, 6z_1, -2z_4 - 3 z_5, z_4, z_5) \Rightarrow\)   \{(1,6,0,0,0), (0,0,-2,1,0), (0,0,-3,0,1)\}
    \item \(\{(1,6,0,0,0), (0,0,-2,1,0), (0,0,-3,0,1), (0,1,0,0,0), (0,0,1,0,0)\}\)
    \item \(\Wcal = \text{span}(\{(0,1,0,0,0), (0,0,1,0,0)\})\)
\end{enumerate}
\end{proof}

\begin{problem}{5}
Suppose \(V\) is finite-dimensional and \(\Ucal, \Wcal\) are subspaces of \(V\) such that \(V = \Ucal
+ \Wcal\). Prove that there exists a basis of \(V\) consisting of vectors in \(\Ucal \cup \Wcal\).    
\end{problem}

\begin{proof}
Let \(\{v_i\}_{i=1}^m\) denote the basis for the vector space \(V\). By definition we have 
\(v_i = u_i + w_i\) for some \(u_i, w_i\). Then we have the spanning set of the vector space \(V\)
\(\sum_i^m a_i (u_i + w_i)\), which can be reduced to a basis by the lemma.
\end{proof}

\begin{problem}{7}
    Suppose \(v_1, v_2, v_3, v_4\) is a basis of \(V\). Prove that 
    \[v_1 + v_2, v_2 + v_3, v_3+v_4, v_4\]
    is also a basis of \(V\). 
\end{problem}

\begin{proof}
We know \(v_1, v_2, v_3, v_4\) is linearly independent and spans \(V\). 
\[a_1(v_1 + v_2) + a_2 (v_2 + v_3) + a_3 (v_3 + v_4) + a_4 v_4 = a_1 v_1 
+ (a_1 + a_2) v_2 + (a_2 + a_3)v_3 + (a_3 + a_4)v_4\]
which shows the linear independence. For proving spanning, let \(v \in V\) then 

\[v = \sum_{i=1}^4 a_i v_i = a_1(v_1 + v_2) + (a_2 - a_1)(v_2 + v_3)
+ (a_3 - a_2)(v_3 + v_4) + (a_4 - a_3)v_4\]
\end{proof}

\begin{problem}{8}
    Prove or give a counterexample: If \(v_1, v_2, v_3, v_4\) is a basis of \(V\) and 
    \(\Ucal\) is a subspace of \(V\) such that \(v_1, v_2 \in \Ucal\) and \(v_3 \notin \Ucal\)
    and \(v_4 \notin \Ucal\), then \(v_1, v_2\) is a basis of \(\Ucal\). 
\end{problem}

\begin{proof}
Take \(V = \R^4\) and the standard basis. Consider \(\Ucal = \{(x_1, x_2, x_3, kx_3)\}\), then 
we disprove the claim. 
\end{proof}

\begin{problem}{10}
    Suppose \(\Ucal\) and \(\Wcal\) are subspaces of \(V\) s.t. \(V = \Ucal \oplus \Wcal\). Suppose 
    also that \(u_1, \ldots u_m\) is a basis of \(\Ucal\) and \(w_1, \ldots, w_n\) is a basis of 
    \(\Wcal\). Prove that 
    \[u_1, \ldots, u_m, w_1, \ldots, w_n\]
    is a basis of \(V\). 
\end{problem}

\begin{proof}
We know 
that this set is linearly independent (otherwise violating the direct sum assumption) 
so it sufficies to prove the spanning. Let \(v \in V\),
then \(v = u + w = \sum_{i=1}^m a_i u_i + \sum_{j=1}^n b_j w_j\). 
\end{proof}

\newpage 

\section*{2C: Dimension}
\addcontentsline{toc}{section}{2C: Dimension}

\begin{lemma}[basis length does not depend on basis]
    Any two bases of a finite-dimensional vector space have the same length. 
\end{lemma}

\begin{definition}[dimension]
    \begin{itemize}
        \item The \emph{dimension} of a finite-dimensional vector space is the length of 
        any basis of the vector space. 
        \item The dimension of a finite-dimensional vector space \(V\) is denoted by dim \(V\).
    \end{itemize}
\end{definition}

\begin{corollary}
    If \(V\) is finite-dimensional and \(\Ucal\) is a subspace of \(V\), then 
    \(\dim \Ucal \leq \dim V\). 
\end{corollary}

\begin{corollary}
    Suppose \(V\) is finite-dimensional. Then every linearly independent list of vectors 
    in \(V\) of length \(\dim V\) is a basis of \(V\). 
\end{corollary}

\begin{corollary}
    Suppose that \(V\) is finite-dimensional and \(\Ucal\) is a subspace of \(V\) such that 
    \(\dim \Ucal = \dim V\). Then \(\Ucal = V\).
\end{corollary}

\begin{corollary}
    Suppose that \(V\) is finite-dimensional. Then every spanning list of vectors in \(V\)
    of length dim \(V\) is a basis of \(V\). 
\end{corollary}

\begin{thm}[dimension of a sum]
    If \(V_1\) and \(V_2\) are subspaces of a finite-dimensional vector space, then 
    \[\dim(V_1 + V_2) = \dim V_1 + \dim V_ 2 - \dim (V_1 \cap V_2)\]
\end{thm}

\newpage 
\addcontentsline{toc}{subsection}{2C Problem Sets}

\begin{problem}{1}
    Show that the subspaces of \(\R^2\) are precisely \(\{0\}\), all lines in \(\R^2\)
    containing the origin and \(\R^2\).
\end{problem}

\begin{proof}
We know \(\dim (\R^2) = 2\) so the subspace dimension is either 0 (\(\{0\}\)) or 
1 (then this means it has to be lines crossing the origin). 
\end{proof}

\begin{problem}{5}
    (a) Let \(\Ucal = \{p \in \Pcal_4 (\F) \colon p(2) = p(5)\}\). Find a basis of \(\Ucal\).

    (b) Extend the basis in (a) to a basis of \(\Pcal_4 (\F)\). 

    (c) Find a subspace \(\Wcal\) of \(\Pcal_4 (\F)\) s.t. \(\Pcal_4(\F) = \Ucal \oplus \Wcal\).
\end{problem}

\begin{proof}
(a)  This means that 
\[a_0 + 2a_1 + 4a_2 + 8a_3 + 16a_4 = a_0 + 5a_1 + 25a_2 + 125 a_3 + 625a_4\]

Solving this gives that \(a_1 = -7a_2 - 39a_3 - 203 a_4\). So we can write that 

\begin{align*}
    p(x)
    &= a_0 + (-7a_2 - 39a_3 - 203 a_4)x + a_2 x^2 + a_3 x^3 + a_4 x^4 \\ 
    &= a_0 + a_2(x^2 - 7x) + a_3(x^3 - 39x) + a_4 (x^4 - 203x)
\end{align*}

The basis now becomes \(\{1, x^2 - 7x, x^3 - 39x, x^4 - 203x\}\)

(b) \(\{1, x, x^2 - 7x, x^3 - 39x, x^4 - 203x\}\)

(c) \(\Wcal = \{x\}\)
\end{proof}

\begin{problem}{8}
    Suppose \(v_1, \ldots, v_m\) is linearly independent in \(V\) and \(w \in V\). Prove that 

    \[\dim \text{span} (v_1 + w, \ldots, v_m + w) \geq m - 1\]
\end{problem}

\begin{proof}
We claim that 
\[v_{i+1} - v_i \in \text{span}(v_1 + w, \ldots, v_m + w) \text{ for all } m \geq i \geq 2\]

as \(v_{i+1} - v_i = (v_{i+1} + w) - (v_i + w)\). Thus \(v_2 - v_1, \ldots, v_m - v_{m-1}\) is 
in the \(\text{span}(v_1 + w, \ldots, v_m + w)\). We've proved that \(v_2 - v_1, \ldots, v_m - v_{m-1}\)
is linearly independent and this list has \(m - 1\) vectors and thus we've proved the claim.
\end{proof}

\begin{problem}{9}
    Suppose \(m\) is a positive integer and \(p_0, p_1, \ldots, p_m \in \Pcal(\F)\) are 
    such that each \(p_k\) has degree \(k\). Prove that \(p_0, p_1, \ldots, p_m\) 
    is a basis of \(\Pcal_m (\F)\).
\end{problem}

\begin{proof}
It's easy to see that this list is linearly independent. Take any element in \(\Pcal_m (\F)\),
we can decompose that by degrees and get each component to be some multiple of \(p_i's\).
\end{proof}

\begin{problem}{10}
    Suppose \(m\) is a positive integer. For \(0 \leq k \leq m\), let 
    \[p_k(x) = x^k (1 - x)^{m - k}\]
    Show that \(p_0, \ldots, p_m\) is a basis of \(\Pcal_m(\F)\). 
\end{problem}

\begin{proof}
It suffices to prove that \(p_k(x)\) are linearly independent. We know from binomial theorem 
that 
\[(x+y)^m = \sum_{i=0}^{m} \binom{m}{i} x^i y^{m-i} \]

Applying this identity here we get that 
\[p_k(x) = x^k \sum_{i=0}^{m-k} \binom{m-k}{i} 1^{m-i} (-1)^{i} x^{i} = \sum_{i=0}^{m-k}
a_i x^{k + i}\]

where \(a_i = \binom{m-k}{i} (-1)^i\). For the polynomial to be identically 0 (\(\sum_{k=0}^{m}c_k p_k (x) = 0\)), 
for each power \(x^i\) we need the coefficient to be 0:

\[\sum_{k=0}^{\min(i, m)} c_k \binom{m-k}{i-k} (-1)^{i - k} = 0\]

We can prove the only solution for this is all \(c_k = 0\) by induction on \(m\). 

The base case is trivial. Assume the statement holds for \(k = m - 1\), then we try 
to prove for \(p_m(x)\), where we have \(P(x) - c_m p_m (x) = \sum_{k=0}^{m-1}c_kp_k(x)=0\). 
This means that \(c_m = 0\) and we've proved the linear independence. 
\end{proof}

\begin{problem}{11}
    Suppose \(\Ucal\) and \(\Wcal\) are both four-dimensional subspaces of \(\C^6\). Prove that 
    there exist two vectors in \(\Ucal \cap \Wcal\) such that neither of these vectors is a 
    scalar multiple of the other. 
\end{problem}

\begin{proof}
\(\dim (\Ucal \cap \Wcal) = \dim(\Ucal) + \dim (\Wcal) - \dim(\Ucal + \Wcal) \geq 8-6= 2\)

So there must exist two linearly independent vectors in the intersection. 
\end{proof}

\begin{problem}{12}
    Suppose that \(\Ucal\) and \(\Wcal\) are subspaces of \(\R^8\) such that 
    \(\dim \Ucal = 3, \dim \Wcal = 5\) and \(\Ucal + \Wcal = \R^8\). Prove that 
    \(\R^8 = \Ucal \oplus \Wcal\).
\end{problem}

\begin{proof}
Similar to problem 11, we can get that \(\dim (\Ucal \cap \Wcal) = 0\). 
\end{proof}

\begin{problem}{14}
    Suppose \(V\) is a ten-dimensional vector space and \(V_1, V_2, V_3\) are subspaces 
    of \(V\) with \(\dim V_1 = \dim V_2 = \dim V_3 = 7\). Prove that \(V_1 \cap V_2 
    \cap V_3 \neq \{0\}\). 
\end{problem}

\begin{proof}
\[\dim ((V_1 \cap V_2) + V_3) = \dim (V_1 \cap V_2) + \dim V_3 - \dim (V_1 \cap V_2 \cap V_3) \]
and also that 
\[\dim (V_1 + V_2) = \dim V_1 + \dim V_2 - \dim(V_1 \cap V_2)\]

This gives that 

\[\dim (V_1 \cap V_2 \cap V_3) = \dim V_1 + \dim V_2 + \dim V_3 - \dim (V_1 + V_2) - 
\dim ((V_1 \cap V_2) + V_3)\]

Note that from question we know \(\dim V_1 + \dim V_2 + \dim V_3 = 21 > 20 = 2\dim V\)

Hence we know 

\[\dim (V_1 \cap V_2 \cap V_3) > (\dim(V) - \dim(V_1 + V_2)) + (\dim(V) - \dim(
    (V_1 \cap V_2) + V_3
)) > 0\]
We've thus proved the claim.
\end{proof}

\begin{problem}{16}
    Suppose \(V\) is finite-dimensional and \(\Ucal\) is a subspace of \(V\)
    with \(\Ucal \neq V\). Let \(n = \dim V\) and \(m = \dim \Ucal\). Prove that 
    there exist \(n - m\) subspaces of \(V\), each of dimension \(n -1\), whose 
    intersection equals \(\Ucal\). 
\end{problem}

\begin{proof}
    To show existence, we can start with the basis for \(\Ucal: u_1, \ldots, u_m\). 
    We extend the basis to \(V: \{ u_1, \ldots, u_m, v_1, \ldots, v_{n-m}\} \coloneqq K\). Construct 
    the subspace \(V_i = \text{span}\{K \setminus \{v_i\}\}\). Then we know that 
    \(\bigcap_i V_i = \text{span}\{\{u_1, \ldots, u_m\}\}\). 
\end{proof}

\begin{problem}{17}
    Suppose that \(V_1, \ldots, V_m\) are finite-dimensional subspaces of \(V\). Prove that 
    \(V_1 + \cdots + V_m\) is finite-dimensional and 
    \[\dim(V_1 + \cdots + V_m) \leq \dim V_1 + \cdots + \dim V_m\]
\end{problem}

\begin{proof}
We prove by induction on \(m\). The base case is trivial. Assume the statement hold for \(k\),
then for \(k+1\), we have that (Denote \(V_1 + \ldots + V_k = M_k\))

\begin{align*}
    \dim(M_k + V_{k+1}) \leq \dim(V_1) + \cdots +\dim (V_{k+1}) 
\end{align*}

which is finite.
\end{proof}

\begin{problem}{18}
    Suppose \(V\) is finite-dimensional with \(\dim V = n \geq 1\). Prove that there exist 
    one-dimensional subspaces \(V_1, \ldots, V_n\) of \(V\) such that 
    \[V = V_1 \oplus \cdots V_n\]
\end{problem}

\begin{proof}
We know there are basis \(\{v_1, \ldots, v_n\}\) for \(V\). Hence we can construct 
\[V_i = \text{span}\{v_i\}\]
\end{proof}

\begin{problem}{19}
    Prove or give a counter example:

    \begin{align*}
        \dim(V_1 + V_2 + V_3) = &\dim V_1 + \dim V_2 + \dim V_3\\ 
        &-\dim(V_1 \cap V_2) - \dim (V_1 \cap V_3) - \dim (V_2 \cap V_3) \\ 
        &+\dim(V_1 \cap V_2 \cap V_3)
    \end{align*}
\end{problem}

\begin{proof}
    Consider $V_{1} = \langle e_{1} \rangle, V_{2} = \langle e_{2} \rangle, V_{3} = \langle e_{1}+e_{2} \rangle $ \\ \\
    then L.H.S = 2 ,but R.H.S = 3.
\end{proof}
\begin{problem}{20}
    Prove that if \( V_1, V_2 \), and \( V_3 \) are subspaces of a finite-dimensional vector space, then

\[
\dim(V_1 + V_2 + V_3)
\]

\[
= \dim V_1 + \dim V_2 + \dim V_3
\]

\[
- \frac{\dim(V_1 \cap V_2) + \dim(V_1 \cap V_3) + \dim(V_2 \cap V_3)}{3}
\]

\[
- \frac{\dim((V_1 + V_2) \cap V_3) + \dim((V_1 + V_3) \cap V_2) + \dim((V_2 + V_3) \cap V_1)}{3}.
\]

\end{problem}

\begin{proof}
We know that 

\begin{align*}
&\dim ((V_1 + V_2) + V_3) = \dim(V_1 + V_2) + \dim(V_3) - \dim((V_1 + V_2)\cap V_3) \\ 
&= \dim (V_1) + \dim (V_2) + \dim (V_3) - \dim(V_1 \cap V_2) - \dim((V_1 + V_2)\cap V_3)
\end{align*}\\

then, by considering $3\dim(V_{1}+V_{2}+V_{3})$,
and using above equality by pairing $V_{1},V_{2},V_{3}$, we obtain the desired equality.



\end{proof}

\end{document}